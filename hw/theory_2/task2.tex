\begin{gather*}
	p(x|y) = p(y|x) p(x) / p(y)
\end{gather*}

Также как и в ттретьей задаче получаем, что выражение в скобках под экспонентой является квадратичной формой относительно x, при этом 

\begin{gather*}
	K = (\Sigma^{-1}+A^T\Gamma^{-1}A)^{-1} \\
	k = (\Sigma^{-1}+A^T\Gamma^{-1}A)^{-1} (A^T \Gamma^{-1}y + \Sigma^{-1}\mu)
\end{gather*}

И соответственно 

\begin{gather*}
	p(x|y) = \mathcal{N} (x|k, K) = \\
	\mathcal{N} (x|(\Sigma^{-1}+A^T\Gamma^{-1}A)^{-1} (A^T \Gamma^{-1}y + \Sigma^{-1}\mu), (\Sigma^{-1}+A^T\Gamma^{-1}A)^{-1})
\end{gather*}