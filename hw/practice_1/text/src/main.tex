\documentclass[unicode]{article}

\usepackage[russian]{babel}
\usepackage[utf8]{inputenc}
\usepackage{color}
\usepackage{amsmath}
\usepackage{amssymb}
\usepackage{epsfig}
\usepackage[colorlinks,urlcolor=blue]{hyperref}
\usepackage{import}
\usepackage{graphicx}
\usepackage{float}

\graphicspath{ {../../imgs/} }


\textheight=24cm
\textwidth=18cm
\oddsidemargin=-1cm
\topmargin=-2.5cm

\newcounter{example}

\def\vec#1{\mathchoice{\mbox{\boldmath$\displaystyle#1$}}
{\mbox{\boldmath$\textstyle#1$}} {\mbox{\boldmath$\scriptstyle#1$}} {\mbox{\boldmath$\scriptscriptstyle#1$}}}

\newcommand{\tr}{\mathop{\mathrm{tr}}\limits}

\DeclareMathOperator{\R}{U}
\DeclareMathOperator{\B}{Beta}
\DeclareMathOperator{\Par}{Pareto}

\renewcommand{\labelitemi}{$-$}

\pagestyle{empty}

\begin{document}
	Домашнее задание по курсу "Байесовские методы в машинном обучении", EM алгоритм для поиска преступника. 
	
	Студент: Юсов Александр
	\section{Вывод формул}
	
		\subsection{E-шаг}
		\import{./}{e_step.tex}
		
		\subsection{M-шаг}
		\import{./}{m_step_base.tex}
		
	\section{Анализ полученных результатов}
		
		\subsection{Влияние начального приближения на результаты}
		\import{./}{analysis_1.tex}
		
		\subsection{Анализ алгоритма при разном уровне шума}
		\import{./}{analysis_2.tex}
		
		\subsection{Поиск лица подозреваемого}
		\import{./}{analysis_3.tex}
		
		\subsection{Сравнение качества EM и Hard-EM алгоритма}
		\import{./}{analysis_4.tex}
		
		\subsection{Улучшения алгоритма}
		\import{./}{analysis_5.tex}
\end{document}