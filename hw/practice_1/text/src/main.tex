\documentclass[unicode]{article}

\usepackage[russian]{babel}
\usepackage[utf8]{inputenc}
\usepackage{color}
\usepackage{amsmath}
\usepackage{amssymb}
\usepackage{graphics}
\usepackage{epsfig}
\usepackage[colorlinks,urlcolor=blue]{hyperref}
\usepackage{import}

\textheight=24cm
\textwidth=18cm
\oddsidemargin=-1cm
\topmargin=-2.5cm
\sloppy

\newcounter{example}

\def\vec#1{\mathchoice{\mbox{\boldmath$\displaystyle#1$}}
{\mbox{\boldmath$\textstyle#1$}} {\mbox{\boldmath$\scriptstyle#1$}} {\mbox{\boldmath$\scriptscriptstyle#1$}}}

\newcommand{\tr}{\mathop{\mathrm{tr}}\limits}

\DeclareMathOperator{\R}{U}
\DeclareMathOperator{\B}{Beta}
\DeclareMathOperator{\Par}{Pareto}

\renewcommand{\labelitemi}{$-$}

\pagestyle{empty}

\begin{document}
	E-шаг:
	\import{./}{update_q.tex}
	
	M-шаг:
	
	Правило обновления априорного распределения на положение лица подозреваемого
	\import{./}{update_A.tex}
	
	Правило обновления матрицы средних для лица подозреваемого
	\import{./}{update_F.tex}
	
	Правило обновления матрицы средних для фона
	\import{./}{update_B.tex}
	
	Правило обновления s
	\import{./}{update_s.tex}
\end{document}