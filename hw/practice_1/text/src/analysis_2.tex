В данном пункте анализа работы алгоритма сгенерируем выборки с различными уровнем зашумления. В данном пункте было решено изменить картинку для лица в целях наглядности

\begin{figure}[H]
	\label{smile_fg}
	\includegraphics[width=6cm, height=6cm]{foreground6.png}
	\centering
	\caption{Картинка, использованная в качестве лица в данном эксперименте}
\end{figure}

В результате для различных уровней шума получаем следующие примеры картинок из датасета:

\begin{figure}[H]
	\label{noise_images_samples}
	\includegraphics[width=\textwidth, height=3cm]{noise_images_samples.png}
	\centering
	\caption{Примеры картинок с разным уровнем шума}
\end{figure}

Для некольких уровней шума и нескольких размеров выборки был запущен EM алгоритм с тремя перезапусками и параметром max\_iter = 10 

\begin{figure}[H]
	\label{estimations_noise_samples}
	\includegraphics[width=\textwidth, height=15cm]{estimations_noise_samples.png}
	\centering
	\caption{Результаты работы алгоритма для датасетов с разными уровнями шума и размерами выборки}
\end{figure}

Видно, что чем больше элементов в выборке, тем проще алгоритму распознать лица на зашумлённых картинках. Однако, возможно, в данном случае 10 итераций было не достаточно и результаты работы алгоритма для больших выборок могли быть лучше.